\documentclass[final,hyperref={pdfpagelabels=false}]{beamer}

\usepackage{amsmath,amsfonts,amsthm,graphicx,harvard,comment}

% \usepackage[absolute,overlay]{textpos}

% \usepackage[orientation=landscape,size=custom,width=90,height=51, debug]{beamerposter}
% \usepackage[orientation=landscape,size=custom,width=60,height=34, debug]{beamerposter}

% Abt SRBI color theme
\usecolortheme{bullfinch}

\DeclareMathSymbol{\niceV}{\mathalpha}{AMSb}{"56}
\DeclareMathSymbol{\niceE}{\mathalpha}{AMSb}{"45}
\DeclareMathSymbol{\niceB}{\mathalpha}{AMSb}{"42}
% \newcommand{\beq}{\begin{equation}}
% \newcommand{\eeq}{\end{equation}}
\newcommand{\pto}{\stackrel{p}{\longrightarrow}}
\newcommand{\dto}{\stackrel{d}{\longrightarrow}}
\newcommand{\Reals}{{\rm I\kern-.25em R}}
\newcommand{\Cov}{\mathop{\rm Cov}\nolimits}
\newcommand{\Var}{\mathop{\niceV}\nolimits}
\newcommand{\Bias}{\mathop{\niceB}\nolimits}
\newcommand{\Expect}{\mathop{\niceE}\nolimits}
\newcommand{\Expected}{\Expect}
\newcommand{\tr}{\mathop{\rm tr}\nolimits}
\newcommand{\trace}{\tr}
\newcommand{\Prob}{\mathop{\rm Prob}\nolimits}
\newcommand{\sign}{\mathop{\rm sign}\nolimits}
\newcommand{\corr}{\mathop{\rm corr}\nolimits}
\newcommand{\Corr}{\mathop{\rm Corr}\nolimits}
\newcommand{\diag}{\mathop{\rm diag}\nolimits}
\newcommand{\mixzero}{{\frac 12 \chi^2_0 + \frac 12 \chi^2_1}}
\newcommand{\rank}{\mathop{\rm rk}\nolimits}
\newcommand{\bm}[1]{\mbox{\boldmath$#1$}}
\newcommand{\bfx}{{\bf x}}
\newcommand{\One}{{\rm 1\kern-.3em I}}
\renewcommand{\mathbf}[1]{\mbox{\boldmath$#1$}}

\renewcommand{\baselinestretch}{1.3}

% \newtheorem{lemma}{Lemma}

\begin{document}


\title{Sampling with minimal strata size requirements}

\author{Stas Kolenikov, Abt SRBI
\and \\
Igor Griva, George Mason University}
\date{JSM 2016}
\institute{}

% redefine beamer colors

\logo{\includegraphics[height=0.8cm]{abt_srbi_rgb}}

\begin{frame}
\maketitle
\end{frame}

\begin{frame}\frametitle{Problem statement}

\begin{itemize}
\item Finite population $\mathcal U=\bigcup_h \, \mathcal U_h$
\item $H$ strata of sizes $N_h$, $\sum_{h=1}^H N_h = N$
\item Create a sampling design $\mathcal S$:
    \begin{enumerate}
        \item total sample size $\sum_{h=1}^H n_h = n$
        \item minimal strata sizes $m_h$, $n_h \ge m_h \ge 0$
    \end{enumerate}
\end{itemize}

Additional $n-\sum_h m_h \ge 0$ units need to
be distributed across the strata.

\bigskip

A simple generalization: allow for stratum-specific costs $c_h$;
the study size is then determined by the total budget
$\sum_h n_h c_h \le C$.

\end{frame}

\begin{frame}\frametitle{Literature}

\begin{itemize}
\item \citeasnoun{neym:1934} considered sampling with different costs
\item \citeasnoun{choudhry:rao:hidiroglou:2012} consider a similar allocation problem,
resort to black box general numeric optimization
\item \citeasnoun{wright:2012} proposed an interesting connection to elections, presented
a unit-by-unit algorithm
\end{itemize}

\end{frame}

\begin{frame}\frametitle{Contribution}

\begin{itemize}
    \item Full consideration of a constrained optimization problem
    \item Explicit algorithm
    \item Extensions to more complicated designs (dual-frame RDD)
\end{itemize}

\end{frame}

\begin{frame}\frametitle{Example}

\hspace{-0.5cm}
\begin{tabular}{l|c|cc|ccc}
                 & Total pop  & Incidence   & $S_h^2$   & $m_h=0$ & $m_h=20$ & $m_h=100$ \\
        \hline
            CT   & 3,592,053  & 14.28\%     & 0.122 & 69  &  67   & 100 \\
            ME   & 1,328,535  & 1.40\%      & 0.014 & 9   &  20   & 100 \\
            MA   & 6,657,291  & 10.24\%     & 0.092 & 111 &  107  & 100 \\
            NH   & 1,321,069  & 3.05\%      & 0.030 & 13  &  20   & 100 \\
            NJ   & 8,874,374  & 18.59\%     & 0.151 & 190 &  183  & 100 \\
            NY   & 19,594,330 & 18.17\%     & 0.149 & 415 &  400  & 200 \\
            PA   & 12,758,729 & 6.15\%      & 0.058 & 168 &  163  & 100 \\
            RI   & 1,053,252  & 13.28\%     & 0.115 & 20  &  20   & 100 \\
            VT   & 626,358    & 1.63\%      & 0.016 & 5   &  20   & 100 \\
\end{tabular}

\end{frame}

\begin{frame}\frametitle{At the interactive session\ldots}

\begin{itemize}
    \item \ldots the slides will be rolling
    \item \ldots I will walk you through an example in Excel
    \item \ldots copies of the paper can be obtained
\end{itemize}

\end{frame}

\begin{frame}\frametitle{References}
\bibliography{everything}
\bibliographystyle{agsm}
\end{frame}

\end{document}



        \alert<3>{
        To expand on \citeasnoun{neym:1938} optimal design framework,
        let the population variance and unit costs be
        $S_h^2$ and $c_h$ for stratum $h$. If the costs are identical
        within strata, then the overall budget constraint
        is automatically replaced by the total sample size constraint.
        Let us parameterize the stratum sample size as
        \begin{equation}
            \label{eq:stratum:size}
            n_h = m_h + t_h
        \end{equation}
        where $t_h\ge 0$. Let the parameter of interest be the
        population mean $\bar Y = \sum_{i \in \mathcal U} Y_i/N$,
        estimated by
        \begin{equation}
            \bar y_{\rm str} = \sum_h W_h y_h,
            \quad W_h = N_h/N,
            \quad y_h = \sum_{i \in \mathcal S_h} y_i/n_h
        \end{equation}
        }

        \alert<4>{
        Then the sample design problem is
        \begin{align}
            \Var[ \bar y_{\rm str}]
               = & \sum_{h=1}^H W_h^2 \frac{S_h^2}{m_h + t_h}
               \to \min_{\{ t_h \}}
            \label{eq:design:optimization}
            \\
            \mbox{s.t. } & \sum_h c_h (m_h + t_h) = C,
            \label{eq:design:budget}
            \\
            t_h & \ge 0 \mbox{ for all } h
            \label{eq:design:thge0}
            \\
            C & \ge \sum_h c_h m_h
            \label{eq:budget:interesting}
        \end{align}
        for the solution to exist.
        }

        \alert<5>{
        See also \citeasnoun{choudhry:rao:hidiroglou:2012}.
        }
    \end{block}

\end{textblock}

\begin{textblock}{26}(29,8)
    \begin{block}{Computation}
        \begin{enumerate}
            \item<alert@6> Set the convergence criteria $\epsilon$
                  (e.g., $\epsilon = C \cdot 10^{-6}$).
            \item<alert@7> Find the upper bound
                  $\overline \lambda=\max_h \frac{W_h^2 S_h^2}{c_h m_h^2}$.
            \item<alert@8> Find the lower bound
                  $\underline \lambda=\min_h \frac{W_h^2 S_h^2}{c_h m_h^2}$.
            \item<alert@9> If $C(\underline \lambda) \le C$, none of the constraints
                    in (\ref{eq:design:thge0}) are binding, and the optimal
                    allocation is the Neyman-Chuprow allocation.
                    \label{alg:neyman}
            \item<alert@10> Set $\lambda^t \leftarrow \overline \lambda$,
                  $\lambda^b \leftarrow \underline \lambda$,
                  $k \leftarrow 1$.
            \item<alert@11> Set $\lambda^{(k)} \leftarrow (\lambda^b + \lambda^t)/2$.
                  \label{alg:midpoint}
            \item<alert@12> Compute $t_h(\lambda^{(k)})$ where \begin{equation}
                        t_h(\lambda) =
                                \max\Bigl[ \frac{W_h S_h}{\sqrt{\lambda c_h}} - m_h, 0 \Bigr],
                                h=1, \ldots, H
                        \label{eq:th}
                    \end{equation}
            \item<alert@13> Evaluate the budget constraint $C(\lambda^{(k)})$ where
                    \begin{equation}
                        C(\lambda)=\sum_h c_h \bigl[m_h + t_h(\lambda)\bigr]
                        \label{eq:c:lambda}
                    \end{equation}
            \item<alert@14> If $|C-C(\lambda^{(k)})| < \epsilon$,
                    go to step \ref{alg:last}.
            \item<alert@15> If the sample size is too large (over budget),
                  increase $\lambda$:
                  set $\lambda^b \leftarrow \lambda^{(k)}, k \leftarrow k+1$.
            \item<alert@16> If the sample size is too small (under budget), decrease $\lambda$:
                  set $\lambda^t \leftarrow \lambda^{(k)}, k \leftarrow k+1$.
            \item<alert@17> Re-iterate to step \ref{alg:midpoint}.
            \item<alert@18> Set $t_h = t_h(\lambda^{(k)})$, rounding up to
                    the integer part as needed.
                    \label{alg:last}
        \end{enumerate}
    \end{block}

\end{textblock}

\begin{textblock}{26}(85,8)
    \begin{block}{Technicalities in the paper}

    \alert<24>{

        \begin{itemize}
            \item Established that the bounds $\overline \lambda, \underline \lambda$
                contain the Lagrange multiplier for the overall cost constraint
                (\ref{eq:budget:interesting}) of the constrained optimization problem
            \item Establish existence and uniqueness of the solution
            \item Establish convergence of the algorithm
            \item Establish equivalence to Neymann-Chuprow allocation
                when none of the minimal sample size constraints are binding
            \item Extensions for dual frame RDD are considered
        \end{itemize}

    }
    \end{block}
\end{textblock}

\begin{textblock}{26}(57,8)
    \begin{block}{Mock data}
    \alert<19>{
        \begin{tabular}{l|ccc|c}
                 & Total pop  & Hispanic pop & \% Hispanic & S_h^2 \\
        \hline
            CT   & 3,592,053  & 512,795      & 14.28\%     & 0.12238 \\
            ME   & 1,328,535  & 18,592       & 1.40\%      & 0.01380 \\
            MA   & 6,657,291  & 681,824      & 10.24\%     & 0.09193 \\
            NH   & 1,321,069  & 40,301       & 3.05\%      & 0.02958 \\
            NJ   & 8,874,374  & 1,649,784    & 18.59\%     & 0.15134 \\
            NY   & 19,594,330 & 3,559,644    & 18.17\%     & 0.14866 \\
            PA   & 12,758,729 & 784,562      & 6.15\%      & 0.05771 \\
            RI   & 1,053,252  & 139,832      & 13.28\%     & 0.11514 \\
            VT   & 626,358    & 10,226       & 1.63\%      & 0.01606 \\
        \hline
        \end{tabular}
        \smallskip
%        $c_h=1,C=1000$
    }
    \end{block}
\end{textblock}

\begin{textblock}{12}(57,37)
    \begin{block}{$m_h=1,C=1000$}
        \centering

        \alert<20>{

        \begin{tabular}{l|cc}
                 & $t_h(\lambda)$ & $n_h$ \\
        \hline
            CT   & 67.87    & 69  \\
            ME   & 7.55     & 9   \\
            MA   & 109.62   & 111 \\
            NH   & 11.45    & 13  \\
            NJ   & 188.21   & 190 \\
            NY   & 413.06   & 415 \\
            PA   & 166.98   & 168 \\
            RI   & 18.59    & 20  \\
            VT   & 3.35     & 5   \\
        \hline
        \end{tabular}
    \medskip

    $\underline\lambda=2.02\cdot10^{-6}$
    $\lambda = 1.069\cdot10^{-7}$
    $\overline\lambda=0.0183$

    }
    \end{block}
\end{textblock}

\begin{textblock}{12}(71,37)
    \begin{block}{$m_h=20,C=1000$}
        \centering

        \alert<21>{

        \begin{tabular}{l|cc}
                 & $t_h(\lambda)$ & $n_h$ \\
        \hline
            CT   & 46.49   &  67   \\
            ME   & 0       &  20   \\
            MA   & 86.80   &  107  \\
            NH   & 0       &  20   \\
            NJ   & 162.67  &  183  \\
            NY   & 379.73  &  400  \\
            PA   & 142.17  &  163  \\
            RI   & 0       &  20   \\
            VT   & 0       &  20   \\
        \hline
        \end{tabular}
    \smallskip

    $\underline\lambda=5.06\cdot10^{-9}$
    $\lambda = 1.15\cdot10^{-7}$
    $\overline\lambda=4.58\cdot10^{-5}$

    }
    \end{block}
\end{textblock}

\begin{textblock}{12}(85,37)
    \begin{block}{$m_h=50,C=1000$}
        \centering

        \alert<22>{

        \begin{tabular}{l|cc}
                 & $t_h(\lambda)$ & $n_h$ \\
        \hline
            CT   & 7.81     & 58  \\
            ME   & 0        & 50  \\
            MA   & 42.86    & 93  \\
            NH   & 0        & 50  \\
            NJ   & 108.84   & 159 \\
            NY   & 297.58   & 348 \\
            PA   & 91.01    & 142 \\
            RI   & 0        & 50  \\
            VT   & 0        & 50  \\
        \hline
        \end{tabular}
    \smallskip
    $\underline\lambda=8.09\cdot10^{-10}$
    $\lambda = 1.52\cdot10^{-7}$
    $\overline\lambda=7.33\cdot10^{-6}$

    }
    \end{block}
\end{textblock}

\begin{textblock}{12}(99,37)
    \begin{block}{$m_h=100,C=1000$}
        \centering

        \alert<23>{

        \begin{tabular}{l|cc}
                 & $t_h(\lambda)$ & $n_h$ \\
        \hline
            CT   & 0        & 100 \\
            ME   & 0        & 100 \\
            MA   & 0        & 100 \\
            NH   & 0        & 100 \\
            NJ   & 0        & 100 \\
            NY   & 99.61    & 200 \\
            PA   & 0        & 100 \\
            RI   & 0        & 100 \\
            VT   & 0        & 100 \\
        \hline
        \end{tabular}
    \smallskip
    $\underline\lambda=2.02\cdot10^{-10}$
    $\lambda = 4.6\cdot10^{-7}$
    $\overline\lambda=1.83\cdot10^{-6}$

    }
    \end{block}
\end{textblock}

\begin{textblock}{82}(29,72)
    \begin{block}{References}
        \small\alert<25>{
        \bibliography{everything}
        \bibliographystyle{agsm}
        }
    \end{block}
\end{textblock}

\end{frame}

\end{document}
